\section{Diskussion}
Fuer die ideale Betriebstemperatur des CCD-Chips ergab sich hier ein Wert von -15 $^\circ$ C, der allerdings durch die eingeschraenkte Kuehlleistung der Kamera nach unten beschraenkt ist. Wuerde man eine verbesserte Kuehlung verwendet, so waere eine tiefere Temperatur und wahrscheinlich auch ein noch geringerer Dunkelstrom zu erwarten. \\
Der geplottete Verlauf entspricht weitgehend den theoretischen Erwartungen: \\
Die Energie der Elektronen im Halbleitermaterial ist boltzmann-verteilt, geht also mit $exp(-\frac{E}{k \cdot T})$. Der exponentielle Anstieg des Dunkelstroms mit der Temperatur findet sich auch im entsprechenden Plot wieder. \\
Der offset der Exponentialfunktion ist wahrscheinlich auf den bias des CCD-Chips zurueckzufuehren, also den ADU-Wert, der auch bei minimaler Belichtungszeit und geschlossener Blende messbar ist. \\
Bei der Berechnung des Gain-Faktors ergibt sich zwar eine Tendenz bei unterschiedlichen Belichtungszeiten, allerdings ist der Fehler des Gain-Faktors verhaeltnismaessig klein. Fuer eine Verbesserung der Bestimmung des Gain-Faktors duerfte eine genauere Aufschluesselung der Beitraege zur Streuung hilfreich sein, sodass eine genauere Bestimmung von $\sigma^{stat}$ moeglich sein duerfte. \\
Bei der Betrachtung des ADU-Outputs ueber der Intensitaet des eingestrahlten Lichts ergibt sich in guter Naeherung eine lineare Abhaengigkeit. Allerdings sind die ADU-Werte fuer groessere Intensitaten geringer als durch Interpolation des linearen Verlaufs zu erwarten waere. Grund hierfuer kann sein, dass bei grossen Intensitaeten einige Pixel der Aufnahme doch ueberbelichtet sind, sodass hier der ADU-Wert konstant bei dem Maximalwert von etwa 65500 liegt. Des weiteren wurde bei der Auswertung ein Rechteck, das genau die Diode einschloss betrachtet und aufintegriert. Hierbei ist zunaechst der von der von der eingestrahlten Intensitaet unabhaengige Dunkelstrom ein Fehler, der insbesondere bei geringen Helligkeiten eine zunehmend grosse Rolle spielt. Weiter ist es bei den kaum belichteten Aufnahmen schwierig, die Diode exakt \enquote{einzurahmen}, sodass hier unterschiedliche Bereiche betrachtet werden. Dieser Fehler koentte vermieden werden, indem man leichte Stoerungen der Anordnung von Diode und Kamera verhindert und anschliessend immer den gleichen Bildausschnitt betrachtet. Eine weitere Fehlerquelle koennen auch die Filter sein, die moeglicherweise nicht exakt den angegebenen Absorptionsgrad aufweisen. \\
Eventuell koennte die Messung verbessert werden, indem man stat einer Diode mit grossem Helligkeitsgradienten eine homogen ausgeleuchtete Flaeche verwendet, da hier ueberbelichtete Pixel verhindert werden koennen. Des weiteren waere es sinnvoll, von den Diodenaufnahmen einen passenden dark frame zu subtrahieren, sodass der Dunkelstrom keinen Fehler mehr liefert. \\
Tatsaechlich liefert der Versuch aber das erwartete Ergebnis. 