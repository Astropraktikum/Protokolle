\section{Diskussion}
Für die ideale Betriebstemperatur des CCD-Chips ergab sich hier ein Wert von -15 $^\circ$C, der allerdings durch die eingeschränkte Kühlleistung der Kamera nach unten beschränkt ist. Würde man eine verbesserte Kühlung verwendet, so wäre eine tiefere Temperatur und wahrscheinlich auch ein noch geringerer Dunkelstrom zu erwarten. \\
Der geplottete Verlauf entspricht weitgehend den theoretischen Erwartungen: \\
Die Elektronen im Halbleitermaterial sind Boltzmann-verteilt, die Anzahl pro Energie und Temperatur geht also mit $\mathrm{e}^{(-\frac{E}{k \cdot T})}$. Der exponentielle Anstieg des Dunkelstroms mit der Temperatur findet sich auch im entsprechenden Plot wieder. \\
Der offset der Exponentialfunktion ist wahrscheinlich auf den bias des CCD-Chips zurückzuführen, also den ADU-Wert, der auch bei minimaler Belichtungszeit und geschlossener Blende messbar ist. \\
Bei der Berechnung des Gain-Faktors ergibt sich zwar eine Tendenz bei unterschiedlichen Belichtungszeiten, allerdings ist der Fehler des Gain-Faktors verhältnismässig klein. Für eine Verbesserung der Bestimmung des Gain-Faktors dürfte eine genauere Aufschlüsselung der Beiträge zur Streuung hilfreich sein, sodass eine genauere Bestimmung von $\sigma^{stat}$ möglich sein dürfte. \\
Bei der Betrachtung des ADU-Outputs über der Intensität des eingestrahlten Lichts ergibt sich in guter Näherung eine lineare Abhängigkeit. Allerdings sind die ADU-Werte für grössere Intensitäten geringer als durch Interpolation des linearen Verlaufs zu erwarten wäre. Grund hierfür kann sein, dass bei großen Intensitäten einige Pixel der Aufnahme doch überbelichtet sind, sodass hier der ADU-Wert konstant bei dem Maximalwert von etwa 65500 liegt. Des weiteren wurde bei der Auswertung ein Rechteck, das genau die Diode einschloss, betrachtet und aufintegriert. Hierbei ist zunächst der von der von der eingestrahlten Intensität unabhängige Dunkelstrom ein Fehler, der insbesondere bei geringen Helligkeiten eine zunehmend große Rolle spielt. Weiter ist es bei den kaum belichteten Aufnahmen schwierig, die Diode exakt \enquote{einzurahmen}, sodass hier unterschiedliche Bereiche betrachtet werden. Dieser Fehler könnte vermieden werden, indem man leichte Störungen der Anordnung von Diode und Kamera verhindert und anschließend immer den gleichen Bildausschnitt betrachtet. Eine weitere Fehlerquelle können auch die Filter sein, die möglicherweise nicht exakt den angegebenen Absorptionsgrad aufweisen. \\
Eventuell könnte die Messung verbessert werden, indem man statt einer Diode mit großem Helligkeitsgradienten eine homogen ausgeleuchtete Fläche verwendet, da hier überbelichtete Pixel verhindert werden können. Des weiteren wäre es sinnvoll, von den Diodenaufnahmen einen passenden dark frame zu subtrahieren, sodass der Dunkelstrom keinen Fehler mehr liefert. \\
Tatsächlich liefert der Versuch aber das erwartete Ergebnis. 