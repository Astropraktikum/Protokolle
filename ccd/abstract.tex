\section{Abstract}
Zur Untersuchung der Eigenschaften optischer CCDs wurden der Zusammenhang zwischen Dunkelstrom und Temperatur des CCD-Chips, der Gain-Faktor der verwendeten Kamera sowie der Zusammenhang zwischen Lichteinfall und ADU-Output überprüft, daneben wurden bias frames aufgenommen und die Effekte \enquote{smear} und \enquote{blooming} untersucht. Es konnte näherungsweise ein exponentieller Zusammenhang zwischen Dunkelstrom und Temperatur, sowie ein kaum von der Belichtungszeit abhängiger Gain-Faktor $\bar{g} = 1.94 \pm 0.04 \ \frac{\mathrm{adu}}{\mathrm{e}}$ beobachtet werden. Der Zusammenhang zwischen Lichteinfall und ADU-Output ergab sich in Näherung als linear, allerdings sichtbar fehlerbehaftet. 

\newpage
\section{Einleitung}
Zu Beginn der Photographie im 19. Jahrhundert wurden Photoplatten verwendet, um optische Eindrücke festzuhalten. Diese waren mit Silbernitrat beschichtet und verfärbten sich je nach Stärke des Lichteinfalls unterschiedlich dunkel, desweiteren waren aufwendige chemische Prozesse notwendig, um eine weitere Verdunklung bei Lichteinfall nach der gewünschten Belichtungszeit zu verhindern. Im Laufe der zweiten Hälfte des 20. Jahrhunderts wurde dieses Prinzip von digitalen Aufnahmemethoden abgelöst, die insbesondere den Vorteil bieten, dass sie im Gegensatz zu den teuren Photoplatten mehrmals verwendbar sind. \\
Ein Beispiel für solche Technologie ist der CCD-Chip, der heute neben den CMOS-Elementen für digitale Aufnahmen verwendet wird. Ein solcher CCD-Chip besteht aus Halbleiterelementen, in denen bei Lichteinfall Ladung ausgelöst wird, die anschließend ausgelesen und in ein digitales Signal umgewandelt werden kann. \\
In diesem Versuch sollen die Eigenschaften von CCDs näher untersucht werden. Insbesondere wird dabei auf gängige Methoden zur Bildverbesserung sowie auf Effekte, die sich aus dem Ausleseprozess ergeben, eingegangen. 