\section{Methoden}
\subsection{Halbleiter}
Halbleiter sind besondere Elemente mit vier Valenzelektronen, das bekannteste Beispiel hierfür ist der Halbleiter Silizium. Sie können durch Einbringen von Fremdatomen mit drei (p) oder fünf Valenzelektronen (n)  \enquote{dotiert} werden. Die hat zur Folge, dass der dotierte Halbleiter zwar elektrisch neutral ist, sich allerdings Unregelmäßigkeiten im Metallgitter ausbilden, sodass ein
Wird ein Atom mit fünf Valenzelektronen eingebracht, so exisiert im regelmäßigen Siliziumgitter kein freier Platz für das Elektron, sodass es \\
HIER FEHLT NOCH ZEUG

\subsection{Aufbau und Funktion eines CCD-Chips}
Ein CCD-Chip setzt sich aus einer rechtecksgitterförmigen Anordnung von Halbleiterelementen, sogenannten Pixeln, zusammen. An diesen liegt eine Spannung in Sperrrichtung an, sodass zunächst kein Strom fließen kann. Wird der Halbleiter allerdings mit Photonen hinreichend großer Energie bestrahlt, so kann das Photon ein oder mehrere Paare von Elektronen und Löchern aus dem Gitter lösen, die schließlich in einem Potenzialtopf gesammelt werden. \\
Die einzelnen Pixel sind spaltenweise gegeneinander isoliert, allerdings nicht zeilenweise. Durch Anlegen einer zeitlich gepulsten Spannung ($\Phi$-Puls) an die Spalten werden die in den Potenzialtöpfen während der Beleuchtung des Chips gesammelten Elektronen während dem Ausleseprozess Zeile für Zeile in Ausleserichtung verschoben. Am Rand des Gitters befindet sich in Ausleserichtung die Ausleseelektronik, die die Ladungsmenge pro Pixel misst und digitalisiert. Durch den Ausleseprozess sowie aufgrund von Fehlern in der Struktur des CCD-Chips entstehen verschiedene Bildfehler, die im Folgenden dargestellt werden. 

\subsection{Bildfehler und -korrekturen}
Die folgenden Informationen wurden entnommen aus \enquote{Richard Berry, James Burnell, The Handbook of Astronomical Image Processing, Willman-Bell 2006}
CCD-Kameras zählen die auf den Chip auftreffenden Photonen und rechnen diese in ein Signal um. Dabei ist jedoch zunächst zu beachten, dass Photonen nicht gleichmäßig auf den Chip auftreffen, sondern zufällig verteilt. Dabei ergibt sich f"ur die Anzahl der detektierten Photonen pro Zeiteinheit eine Poisson-Verteilung. Bei wiederholten Messungen bekommt man also auch jeweils verschiedene Werte. Dies wäre bei herkömmlichen, z.B. tagsüber aufgenommenen Bildern egal, da dort enorm viele Photonen auftreffen und die kleinen Variationen keinen Unterschied machen. Die Zahl der auftreffenden Photonen bei astronomischen Aufnahmen ist aber so gering, dass die zufälligen Variationen großen Einfluss nehmen k"onnen.\
Bei mehreren Bildern wird für jeden Pixel einen Mittelwert und eine Standardabweichung (auch 'noise' genannt) für die Photonenzahl bestimmt. Anhand dessen wird ein Maß für die Qualität eines Bildes definiert, die \enquote{Signal-to-noise ratio} (SNR) $\frac{\overline{x}}{\sqrt{\overline{x}}} = \sqrt{\overline{x}}$.  Daraus lässt sich auch erkennen, dass eine größere Anzahl detektierter Photonen eine bessere Bildqualität liefert.\\
Die reine statistische Varianz der ankommenden Photonen ist aber nicht die einzige Quelle für Fehler bzw. Abweichungen. So ist der von der CCD-Kamera ausgegebene Wert für jeden Pixel nicht einfach die Zahl der detektierten Photonen, sondern ein Wert in einer willk"urlich gew"ahlten Einheit (genannt 'analog-to-digital unit' oder ADU). Der Zusammenhang zwischen den gelösten Elektronen und dem Wert in ADU heißt 'Gain-Faktor' $g$. Es ist insbesondere nicht klar, ob ein 0 detektierte Photonen auch einem Wert von 0 ADU entspricht, so besitzen die meisten Kameras einen Offset (genannt 'bias'), der vom Messwert abgezogen werden muss. Dieser Offset kann bestimmt werden, indem ein 'bias frame', also ein Bild mit minimaler Belichtungszeit aufgenommen wird. Dieser 'bias frame' muss dann vom eigentlichen Bild abgezogen werden, um ein Ergebnis, das proportional zur Anzahl der detektierten Photonen ist, zu erhalten. Ein weiterer Effekt der Signalwandelelektronik ist der 'readout noise', eine konstante Unsicherheit, die ihren Ursprung in den Verstärkerschaltkreisen hat.
\\
Der CCD-Chip selbst hat auch Fehlerquellen. Defekte im Kristallgitter der Halbleiterstruktur führen zu 'hot pixels' (zu kleine Bandlücke, sodass immer Elektronen herausgelöst werden), 'dead pixels' (zu große Bandlücke, sodass nie Elektronen herausgelöst werden), außerdem führen thermische Effekte (nicht ausreichende Kühlung) ebenfalls dazu, dass Elektronen gelöst werden, ohne dass ein Photon auf den Chip auftrifft. Die thermischen Effekte und 'hot pixels' können mittels einer 'dark frame'-Aufnahme, also einer Aufnahme mit identischer Belichtungszeit zur eigentlichen Aufnahme, aber ohne Lichteinfall (z.B. mit geschlossenem Shutter), korrigiert werden. Auf dem 'dark frame' sind also nur die Fehler des Chips zu sehen, die dann vom eigentlichen Bild abgezogen werden können. Da die Auslösung der Elektronen durch thermische Effekte aber auch ein Zufallsprozess ist, kommt durch die 'dark frame'-Aufnahme eine weitere Unsicherheit hinzu ('dark current noise').\
'Dead pixels' können in der Nachbearbeitung behoben werden, indem der Wert des defekten Pixels als Mittelwert der beiden benachbarten Pixel berechnet wird.
\\
Der dritte wichtige Faktor, der das Bild verfälschen kann, ist der eigentliche optische Aufbau, (d.h. Teleskop, Okular, Filter, etc.) Reflexionen, Staub, Linsen-/Spiegelfehler usw., sowie die unterschiedliche Empfindlichkeit verschiedener Pixel auf Photoneneinfall (z. B. aufgrund leicht unterschiedlicher Bandl"ucken) erzeugen auf dem Bild sichtbare Artefakte. Um diese zu entfernen, wird ein 'flat field' aufgenommen, ein homogen ausgeleuchtetes Bild. Dieses Bild sollte möglichst hell sein (also eine große signal-to-noise ratio besitzen), um wenig zusätzliche Unsicherheiten in das fertige Bild einzubringen. Außerdem kann noch ein 'flat dark frame' aufgenommen werden, der dann vom flat field abgezogen wird. Die eigentliche Aufnahme (die schon mittels bias und dark frame verbessert wurde) wird schließlich mit dem flat field gewichtet (d.h. dadurch geteilt).
\\
Die vorhin schon angesprochenen statistischen Betrachtungen (signal, noise) haben eine wichtige praktische Bedeutung, die hier noch kurz angesprochen werden soll.
\\
Das gesamte detektierte Signal (in ADU) in der ursprünglichen Aufnahme $S_{raw}$ ist gegeben durch
\begin{equation}
S_{raw} = \frac{x}{g} + \frac{x_d}{g} + b
\end{equation}
wobei $x$ die Anzahl der detektierten Elektronen, $x_d$ der Dunkelstrom, also die Anzahl der durch CCD-Effekte herausgelösten Elektronen, $g$ der Gain-Faktor und $b$ der Bias ist.
\\
Die gesamte Unsicherheit $\sigma_{raw}$ (in ADU) der ursprünglichen Aufnahme berechnet sich durch 
\begin{equation}
\sigma_{raw} = \frac{\sqrt{\sigma^2 + \sigma_d^2 + \sigma_{ron}^2}}{g}
\end{equation}
wobei $\sigma$ die Standardabweichung der detektierten Elektronen, $\sigma_d$ der dark current noise und $\sigma_{ron}$ der readout noise ist.
\\
Diese Formeln gelten entsprechend auch für den dark frame und das flat field.
Die praktische Anwendung zeigt sich, wenn man mehrere Bilder aufnimmt (um ein Mittel zu erzeugen). Die resultierende Signalstärke und Abweichung ergibt sich dann zu:
\begin{equation}
S_{combined} = \frac{N_{raw}S_{raw}}{N_{raw}} - \frac{N_{dark}S_{dark}}{N_{dark}} = S_{raw} - S_{dark}
\end{equation}
\begin{equation}
\sigma_{combined} = \sqrt{\frac{\sigma_{raw}^2}{N_{raw}} + \frac{\sigma_{dark}^2}{N_{dark}}}
\end{equation}
Man sieht, dass die Signalstärke nicht von der Anzahl $N$ der Aufnahmen abhängt, wohl aber die Abweichung. Je mehr Bilder man verwendet, desto kleiner wird die Abweichung, also wird die signal-to-noise ratio größer. Mit diesen Formeln kann man abschätzen, wie viele Aufnahmen angefertigt werden müssen, um eine gewünschte Bildqualität zu erhalten. Eine SNR von 3 liefert ein unklares und stark körniges Bild, während eine SNR von 30 bereits ein scharfes und deutliches Bild ergibt. Bei einer SNR von 100 kann man sogar kleine Details klar erkennen und Nebel- und Staubwolken gut sichtbar und scharf abbilden.
\\
Noch zu bedenken ist, dass durch die flat field - Korrektur zusätzliche Unsicherheit hinzukommt. Der Einfluss auf die resultierende Bildqualität ist wegen der sehr hohen SNR der flat field - Aufnahme jedoch üblicherweise vernachlässigbar, insbesondere anhand der enormen optischen Aufwertung des Bildes durch die Korrektur. Die resultierende SNR berechnet sich aus:
\begin{equation}
SNR_{result} = \frac{1}{\sqrt{\frac{1}{SNR_{image}^2} + \frac{1}{SNR_{flatfield}^2}}}
\end{equation}
Ein letzter, wichtiger Aspekt für astronomische Aufnahmen ist der Einfluss der Lichtverschmutzung, der durch die OSR (\enquote{object-to-sky ratio}) beschrieben wird. Dieser kann in städtischen Gebieten (also auch in Bamberg) sehr groß sein, ist aber für diesen Versuch irrelevant.\\
\enquote{Blooming} und \enquote{Smear} sind zwei weitere Effekte, die bei Aufnahmen mit CCD-Chips auftreten können. 
\enquote{Blooming} bezeichnet den Effekt, dass Pixel im CCD-Chip überbelichtet werden, sodass Elektronen in benachbarte Pixel wandern. Dies führt zu einem hellen Schein um das überbelichtete Pixel herum. \\
Beim sogenannten \enquote{Smear} muss der Shutter deaktiviert sein. Es kommt während des Auslesevorgangs zu weiterer Belichtung der Pixel, sodass in der stark belichteten Spalte ein heller Streifen wahrnehmbar ist. 


\subsection{Der Gain-Faktor}Es gilt: 
Für die Berechnung des Gain-Faktors ergibt sich: \

\begin{equation}
\sigma_{EL} = g \cdot \sigma_{ADU}, 
\label{eins}
\end{equation}
\begin{equation}
N_{EL} = g \cdot N_{ADU}
\end{equation}
und 
\begin{equation}
\sigma_{EL} = \sqrt{N_{EL}}. 
\end{equation}
Aus \ref{eins} ergibt sich: 
\begin{equation}
\sigma_{EL}^2 = N_{EL}, 
\end{equation}
sodass gilt: 
\begin{equation}
N_{EL} = g^2 \cdot \sigma_{ADU}^2 = g \cdot N_{ADU}. 
\end{equation}
Daraus ergibt sich: 
\begin{equation}
g = \frac{N_{ADU}}{\sigma_{ADU}^2}
\end{equation}

Da tats"achlich nur das statistische Rauschen Poisson-verteilt ist, Zur möglichst genauen Berechnung des obigen Terms muss der Offset zwischen der $N_{ADU}$ und $N_{EL}$ bei der Aufnahme eines flat fields, der mittels bias frames ermittelt werden kann, herausgerechnet werden, sodass schließlich eine direkte Proportionalität zwischen $N_{ADU,\ ohneBias}$ und $N_{EL}$ besteht. 
Da anschließend zwei systematisch gleiche Aufnahmen betrachtet werden, schreibt man: 
\begin{equation}
N_{ADU,\ ohneBias} = \frac{N_{ADU}(A) + N_{ADU}(C) - \overline{B_1} - \overline{B_2}}{2}, 
\end{equation}
wobei $\overline{B_1}$ und $\overline{B_2}$ die Mittelwerte zweier bias frames sind, die ebenfalls systematisch gleich sind. \\
Für die Standardabweichung von $N_{ADU, ohneBias}$ ergibt sich: \\
Sowohl für $\sigma(A-C)^2$ als auch für $\sigma(B_1 - B_2)^2$ ergibt sich nach der in der Anleitung vorgeführten Rechnung, dass:
\begin{equation}
\sigma_{ADU,\ flat}^2 \ ^{(stat)} = \frac{\sigma_{ADU}^2(A-C)}{2}
\end{equation}
sowie
\begin{equation}
\sigma_{ADU,\ bias}^2 \ ^{(stat)} = \frac{\sigma_{ADU}^2(B_1-B_2)}{2}.
\end{equation}
Da beide Standardabweichungen statistischer Natur sind, ist ihre Kovarianz Null, sodass gilt: 
\begin{equation}
\sigma_{ADU, \ ohne}^2  \ ^{(stat)}=  \sigma_{ADU,\ flat}^2 \ ^{(stat)} - \sigma_{ADU,\ bias}^2 \ ^{(stat)}.
\end{equation}
Durch Einsetzen ergibt sich nun: 
\begin{align}
g &= \frac{N_{ADU,\ ohneBias}}{\sigma_{ADU, \ ohne}^2  \ ^{(stat)}} = \frac{\frac{N_{ADU}(A) + N_{ADU}(C) - \overline{B_1} - \overline{B_2}}{2}}{\sigma_{ADU,\ flat}^2 \ ^{(stat)} - \sigma_{ADU,\ bias}^2 \ ^{(stat)}} &=\frac{\frac{N_{ADU}(A) + N_{ADU}(C) - \overline{B_1} - \overline{B_2}}{2}}{\frac{\sigma_{ADU}^2(A-C) - \sigma_{ADU}^2(B_1-B_2)}{2}} \nonumber\\
&= \frac{N_{ADU}(A) + N_{ADU}(C) - \overline{B_1} - \overline{B_2}}{\sigma_{ADU}^2(A-C) - \sigma_{ADU}^2(B_1-B_2)}.\
\label{form:gain}
\end{align}
Diese Formel wird zur Berechnung des Gain-Faktors verwendet. 

\subsection{Durchführung}
In diesem Versuch soll zunächst die ideale Betriebstemperatur für den CCD-Chip ermittelt werden, also die Temperatur, bei der der Dunkelstrom minimal ist. Dazu werden verschiedene Temperaturen eingestellt und dann ein dark frame aufgenommen und ausgewertet. Es ist zu beachten, dass sich die Temperatur erst einpendeln muss, bevor eine Messung durchgeführt werden kann. \\
Weiter sollen bias frames und flat fields aufgenommen werden, um hieraus einen Wert für den Gain-Faktor zu bestimmen. Dazu wird mit der CCD-Kamera das Bild eines mit einem Baustrahler weitgehend homogen ausgeleuchteten Blatts Papier an der etwa drei Meter von der Kamera entfernten Tür aufgenommen und ausgewertet. Es ist hier zu beachten, dass das Bild nicht überbelichtet sein darf, da es andernfalls zu unerwünschten Effekten wie etwa blooming kommen kann. \\
Schließlich soll der Zusammenhang zwischen der Intensität des einfallenden Lichts und dem ADU-Output des CCD-Chips untersucht werden, indem Filter verschiedener Stärke vor der CCD-Kamera montiert werden. Für sämtliche, auch durch mehrere Filter erzeugte, Filterstärken wird die Helligkeit einer aufgenommenen LED ausgewertet. Weiter werden die angesprochenen Effekte \enquote{blooming} und \enquote{smear} untersucht. 