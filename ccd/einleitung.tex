\section{Einleitung}
Zu Beginn der Photographie im 19. Jahrhundert wurden Photoplatten verwendet, um optische Eindrücke festzuhalten. Diese waren etwa mit Silbernitrat beschichtet und verfärbten sich je nach Stärke des Lichteinfalls unterschiedlich dunkel, desweiteren waren aufwendige chemische Prozesse notwendig, um eine weitere Verdunklung bei Lichteinfall nach der gewünschten Belichtungszeit zu verhindern. Im Laufe der zweiten Hälfte des 20. Jahrhunderts wurde dieses Prinzip von digitalen Aufnahmemethoden abgelöst, die insbesondere den Vorteil bieten, dass sie im Gegensatz zu den teuren Photoplatten mehrmals verwendbar sind. \\
Ein Beispiel für solche Technologie ist der CCD-Chip, der heute neben dem CMOS-Elementen für digitale Aufnahmen verwendet wird. Ein solcher CCD-Chip besteht aus Halbleiterelementen, in denen bei Lichteinfall Ladung ausgelöst wird, die anschließend ausgelesen und in ein digitales Signal umgewandelt werden kann. \\
In diesem Versuch sollen die Eigenschaften von CCDs näher untersucht werden. Insbesondere wird dabei auf gängige Methoden zur Bildverbesserung sowie auf Effekte, die sich aus dem Ausleseprozess ergeben, eingegangen. 