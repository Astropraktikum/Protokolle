\section{Einleitung}
Bereits in der Schule lernt man, dass Licht aus verschiedenen Farben zusammengesetzt ist. Dies zeigt sich z.B. bei Prismen oder am natürlichen Phänomen des Regenbogens. 1814 hat Joseph von Fraunhofer im Sonnenspektrum schwarze Linien entdeckt.\footnote{\ \cite{ktroskopie}} Er konnte den Ursprung der nach ihm benannten Linien jedoch nicht erklären. Heutzutage ist der Ursprung der Linien bekannt. Die Linien entstehen dadurch, dass die Atome der Sonne nur das Licht bestimmter Frequenzen absorbieren können, was es erlaubt, aus Absorptions- oder Emissionsspektren von Licht auf Eigenschaften der Lichtquelle zu schließen. Das Zerlegen und Analysieren von Spektren wird als Spektroskopie bezeichnet und erlaubt die Erforschung vieler Eigenschaften von Himmelskörpern, wie z.B. die Bestimmung von Radialgeschwindigkeiten oder die Spektralklassifikation von Sternen. Im Folgenden wird sich deshalb mit der Spektroskopie beschäftigt.

%Quelle:   http://de.wikipedia.org/wiki/Spektroskopie

Für den elektromagnetischen Feldstärketensor gilt:
\begin{equation}
F^{\mu \nu} = \partial^{\mu} A^{\nu} - \partial^{\nu} A^{\mu}
\end{equation}
Dabei ist A das 4er-Verktorpotential.
Dies wird als Grundwissen vorausgesetzt und hier deshalb nicht weiter vertieft.