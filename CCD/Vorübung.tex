\section{Aufgabe 1}
Die folgenden Informationen wurden entnommen aus 'The Handbook of Astronomical Image Processing, Richard Berry, James Burnell, Willmann-Bell 2006':
\\
CCD-Kameras zählen die auf den Chip auftreffenden Photonen und rechnen diese in ein Signal um. Dabei ist jedoch zunächst zu beachten, dass Photonen nicht gleichmäßig auf den Chip auftreffen, sondern zufällig verteilt. Wenn man also mehrere Bilder aufnimmt, bekommt man auch mehrere Werte für jeden Pixel. Dies wäre bei herkömmlichen, z.B. tagsüber aufgenommenen Bildern egal, da dort enorm viele Photonen auftreffen und die kleinen Variationen keinen Unterschied machen. Die Zahl der auftreffenden Photonen bei astronomischen Aufnahmen ist aber so gering, dass die zufälligen Variationen großen Einfluss nehmen.\
Man kann also bei mehreren Bildern für jeden Pixel einen Mittelwert und eine Standardabweichung (auch 'noise' genannt) für die Photonenzahl bestimmen. Anhand dessen wird ein Maß für die Qualität eines Bildes definiert, die 'Signal-to-noise ratio' (SNR) $\frac{\overline{x}}{\sqrt{\overline{x}}} = \sqrt{\overline{x}}$.  Daraus lässt sich auch erkennen, dass eine größere Anzahl detektierter Photonen eine bessere Bildqualität liefert.\\
Die reine statistische Varianz der ankommenden Photonen ist aber nicht die einzige Quelle für Fehler bzw. Abweichungen. So ist der von der CCD-Kamera ausgegebene Wert für jeden Pixel nicht einfach die Zahl der detektierten Photonen, sondern ein Wert in einer nicht näher definierten Einheit (genannt 'analog-to-digital unit' oder ADU). Der Zusammenhang zwischen den gelösten Elektronen und dem Wert in ADU heißt 'Gain-Faktor' $g$. Es ist insbesondere nicht klar, ob ein Wert von 0 ADU auch 0 detektierten Photonen entspricht, so besitzen die meisten Kameras einen Offset (genannt 'bias'), der vom Messwert abgezogen werden muss. Dieser Offset kann bestimmt werden, indem ein 'bias frame', also ein Bild mit minimaler Belichtungszeit aufgenommen wird. Dieser 'bias frame' muss dann vom eigentlichen Bild abgezogen werden, um ein korrektes Ergebnis zu erhalten. Ein weiterer Effekt der Signalwandelelektronik ist der 'readout noise', eine konstante Unsicherheit, die ihren Ursprung in den Verstärkerschaltkreisen hat.
\\
Der CCD-Chip selbst ist auch eine Fehlerquelle. Defekte im Kristallgitter der Halbleiterstruktur führen zu 'hot pixels' (zu kleine Bandlücke, sodass immer Elektronen herausgelöst sind), 'dead pixels' (zu große Bandlücke, sodass nie Elektronen herausgelöst werden), außerdem führen thermische Effekte (nicht ausreichende Kühlung) ebenfalls dazu, dass Elektronen gelöst werden, ohne dass ein Photon auf den Chip auftrifft. Die thermischen Effekte und 'hot pixels' können mittels einer 'dark frame'-Aufnahme, also einer Aufnahme mit identischer Belichtungszeit zur eigentlichen Aufnahme, aber ohne Lichteinfall (z.B. mit geschlossenem Shutter), korrigiert werden. Auf dem 'dark frame' sind also nur die Fehler des Chips zu sehen, die dann vom eigentlichen Bild abgezogen werden können. Da die Auslösung der Elektronen durch thermische Effekte aber auch ein Zufallsprozess ist, kommt durch die 'dark frame'-Aufnahme eine weitere Unsicherheit hinzu ('dark current noise').\
'Dead pixels' können in der Nachbearbeitung behoben werden, indem der Wert des defekten Pixels als Mittelwert der beiden benachbarten Pixel berechnet wird.
\\
Der dritte wichtige Faktor, der das Bild verfälschen kann, ist der eigentliche optische Aufbau, (d.h. Teleskop, Okular, Filter, etc.) Reflexionen, Staub usw. erzeugen auf dem Bild sichtbare Artefakte. Um diese zu entfernen, wird ein 'flat field' aufgenommen, ein homogen ausgeleuchtetes Bild. Dieses Bild sollte möglichst hell sein (also eine große signal-to-noise ratio besitzen), um wenig zusätzliche Unsicherheiten in das fertige Bild einzubringen. Außerdem kann noch ein 'flat dark frame' aufgenommen werden, der dann vom flat field abgezogen wird. Die eigentliche Aufnahme (die schon mittels bias und dark frame verbessert wurde) wird schließlich mit dem flat field gewichtet (d.h. dadurch geteilt).
\\
Die vorhin schon angesprochenen statistischen Betrachtungen (signal, noise) haben eine wichtige praktische Bedeutung, die hier noch kurz angesprochen werden soll.
\\
Das gesamte detektierte Signal (in ADU) in der ursprünglichen Aufnahme $S_{raw}$ ist gegeben durch
\begin{equation}
S_{raw} = \frac{x}{g} + \frac{x_d}{g} + b
\end{equation}
wobei $x$ die Anzahl der detektierten Elektronen, $x_d$ der Dunkelstrom, also die Anzahl der durch CCD-Effekte herausgelösten Elektronen, $g$ der Gain-Faktor und $b$ der Bias ist.
\\
Die gesamte Unsicherheit $\sigma_{raw}$ (in ADU) der ursprünglichen Aufnahme berechnet sich durch 
\begin{equation}
\sigma_{raw} = \frac{sqrt{\sigma^2 + \sigma_d^2 + \sigma_{ron}^2}}{g}
\end{equation}
wobei \sigma die Standardabweichung der detektierten Elektronen, \sigma_d der dark current noise und \sigma_{ron} der readout noise ist.
\\
Diese Formeln gelten entsprechend auch für den dark frame und das flat field.
Die praktische Anwendung zeigt sich, wenn man mehrere Bilder aufnimmt (um ein Mittel zu erzeugen). Die resultierende Signalstärke und Abweichung ergibt sich dann zu:
\begin{equation}
S_{combined} = \frac{N_{raw}S_{raw}}{N_{raw}} - \frac{N_{dark}S_{dark}}{N_{dark}} = S_{raw} - S_{dark}
\end{equation}
\begin{equation}
\sigma_{combined} = \sqrt{\frac{\sigma_{raw}^2}{N_{raw}} + \frac{\sigma_{dark}^2}{N_{dark}}}
\end{equation}
Man sieht, dass die Signalstärke nicht von der Anzahl $N$ der Aufnahmen abhängt, wohl aber die Abweichung. Je mehr Bilder man verwendet, desto kleiner wird die Abweichung, also wird die signal-to-noise ratio größer. Mit diesen Formeln kann man abschätzen, wie viele Aufnahmen angefertigt werden müssen, um eine gewünschte Bildqualität zu erhalten. Eine SNR von 3 liefert ein unklares und stark körniges Bild, während eine SNR von 30 bereits ein scharfes und deutliches Bild ergibt. Bei einer SNR von 100 kann man sogar kleine Details klar erkennen und Nebel- und Staubwolken gut sichtbar und scharf abbilden.
\\
Noch zu bedenken ist, dass durch die flat field - Korrektur zusätzliche Unsicherheit hinzukommt. Der Einfluss auf die resultierende Bildqualität ist wegen der sehr hohen SNR der flat field - Aufnahme jedoch üblicherweise vernachlässigbar, insbesondere anhand der enormen optischen Aufwertung des Bildes durch die Korrektur. Die resultierende SNR berechnet sich aus:
\begin{equation}
SNR_{result} = \frac{1}{sqrt{\frac{1}{SNR_{image}^2 + \frac{1}{SNR_{flatfield}^2}}
\end{equation}
Ein letzter, wichtiger Aspekt für astronomische Aufnahmen ist der Einfluss der Lichtverschmutzung, der durch die OSR ('object-to-sky ratio') beschrieben wird. Dieser kann in städtischen Gebieten (also auch in Bamberg) sehr groß sein, ist aber für diesen Versuch irrelevant.