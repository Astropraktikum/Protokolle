\section{Methoden}
\subsection{Beobachtung mit den Teleskopen im Garten}
Mit den Teleskopen im Garten sollten zwei \enquote{interessante} Objekte beobachtet werden. Dabei entschieden wir uns für den Jupiter sowie den Orionnebel und den Mond. Des weiteren sollte der Doppelstern $\eta$ Ori mittels Starhopping gefunden und beobachtet werden. Hierzu wurde ein Apochromat verwendet, der auf der süd-westlichen Teleskopsäule montiert wurde. 
\subsection{Starhopping}
Das Auffinden von Sternen am Nachthimmel wurde mittels des sogenannten \enquote{Starhopping} und Auffindkarten realisiert. Das allgemeine Vorgehen dabei ist, dass man sich zu Beginn einen hellen, eindeutig identifizierbaren Stern in der Nähe des zu beobachtenden Objekts sucht und davon ausgehend Objekte in der Umgebung dieses Sterns identifiziert und sich auf diese Art bis zu dem gewünschten Objekt \enquote{durcharbeitet}.  

\subsection{Beobachtung mit dem 50 cm-Teleskop}
Des weiteren sollte ein Nebel, eine Galaxie oder ein Sternhaufen beobachtet werden. Nachdem die \enquote{Sombrero-Galaxie} M 104, die wir eigentlich ausgesucht hatten, zum Beobachtungszeitpunkt leider eine zu kleine Deklination aufwies, entschieden wir uns für die \enquote{Whirlpool-Galaxie} M 51. \\
Der Grundaufbau war eine CCD-Kamera, die an das 50 cm-Teleskop in der Ostkuppel angeschlossen war. 
Zunächst wurde das Teleskop auf den Stern Merak im Sternbild Großer Wagen fokussiert. Dies geschah mittels einer Blende, die drei paarweise nicht parallele Spalten aufweist, sodass es zu Interferenz des einfallenden Lichts kommt. Der Fokus des Teleskops wurde nun mittels der Handsteuerung so eingestellt, dass die drei durch die Interferenz bedingten, am Bildschirm sichtbaren Geraden einen gemeinsamen Schnittpunkt aufweisen. \\
Da sowohl Merak als auch der zu beobachtende Stern hinreichend weit von der Erde entfernt sind, ist die Einstellung des Fokus' für beide Objekte äquivalent. \\
Anschließend wurden zunächst fünf Bilder mit einer Belichtungszeit von 120 s ohne Farbfilter in der Messapparatur aufgenommen. Danach wurden nochmals je drei Bilder mit rot-, grün bzw. blau-Filter aufgenommen, wobei eine Belichtungszeit von 120 s verwendet wurde. Allerdings wurde das sogenannte \enquote{binning} angewendet, wodurch je vier Pixel zu einem zusammengefasst werden, also deren Intensitäten addiert werden. Da aufgrund der Filter die transmittierte Intensität um etwa den Faktor 3 geringer ist als bei der Aufnahme ohne Filter, gleicht das binning diesen Effekt in etwa aus, da es eine Erhöhung der Intensität um den Faktor 4 zur Folge hat. 

\subsection{Bildbearbeitung mittels dark frames und flat fields}
Ein dark frame ist eine Aufnahme bei geschlossenem Shutter. Dies ist notwendig, da es Materialfehler im CCD-Chip gibt, die sich nur schwer und unter hohem Kostenzuwachs bei der Produktion vermeiden lassen. Diese sind beispielsweise sogenannte \enquote{dead pixels}, d. h. Pixel, bei denen die Bandlücke deutlich größer ist als im idealisierten Halbleiter. In diesen Pixeln können also keine Elektronen gelöst werden, sodass diese im Bild unabhängig von der einfallenden Strahlung als schwarz erscheinen.
Ein weiterer Effekt sind die sogenannten \enquote{hot pixel}. d. h. Pixel, bei denen die Bandlücke deutlich geringer ist als im idealisierten Halbleiter. Aus diesem Grund  werden auch während dem Auslesevorgang durch die thermische Bewegung der Atomrümpfe Elektronen gelöst. Da das Auslesen spaltenweise geschieht, werden alle Pixel hinter dem \enquote{hot pixel} in Ausleserichtung im sich ergebenden Bild hell. Diese Pixelfehler treten auch beim dark frame auf, sodass diese Fehler durch pixelweise Subtraktion der Intensitäten des dark frames von der eigentlichen Aufnahme korrigiert werden können. \\
Eine weitere Korrekturmaßnahme ist das Aufnehmen eines flat fields, d. h. einer Aufnahme bei gleichmäßiger Ausleuchtung des aufgenommenen Sichtfeldes. Die Notwendigkeit dieser Korrektur ergibt sich durch Staubkörner, die sich Strahlengang des Teleskops befinden, sowie durch Helligkeitsunterschiede im Bild, die durch den Ausleseprozess bedingt sind. Staubkörner führen zu dunkleren Bereichen im aufgenommenen Bild, da die Körner das Licht streuen und teilweise absorbieren. Des weiteren kann es während dem Auslesevorgang des CCDs dazu kommen, dass aus verschiedenen Gründen Ladung verloren geht oder weitere erzeugt wird. Da das Auslesen leicht zeitverzögert für verschiedene Bereiche des CCDs stattfinden, führt dies bei der Aufnahme zu Helligkeitsunterschieden in Ausleserichtung. Diese Fehler werden korrigiert, indem die Intensitäten der Aufnahme pixelweise durch die Intensitäten des flat fields dividiert werden. Dadurch werden die durch die angesprochenen Fehler dunkleren Bereiche wieder aufgehellt, da das flat field hier ebenfalls einen geringeren Intensitätswert hat.

\subsection{Bildbearbeitung in Astroart}
Die Bearbeitung der Bilder mit dark frames und flat fields erfolgt in Astroart mit Hilfe einer automatisierten Importfunktion. Der Import wird für jeden Filter separat durchgeführt, dabei werden mehrere Belichtungen zu einem Bild verrechnet. Nun liegen Einzelbilder für jeden Farbkanal und für Belichtungen ohne Farbfilter vor. Die Farbkanäle werden nun durch die Funktion Trichromy zu einem RGB Bild verbunden. Das Farbbild wird dann mit dem Bild ohne Farbbilder multipliziert. Die durch das binning geringere Auflösung des Farbbilds wird durch Skalierung ausgeglichen.

\subsection{Freie Beobachtungen mit dem 40 cm-Teleskops}
Die freie Beobachtung astronomischer Objekte geschah mittels des 40 cm-Teleskop in Ostkuppel und dem Zoomokular mit variabler Brennweite zwischen 8 und 24 mm. Dabei sollten verschiedene interessante Objekte mit ausreichender Helligkeit beobachtet werden.  