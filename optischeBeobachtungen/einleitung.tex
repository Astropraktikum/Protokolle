\section{Einleitung}
Die Beobachtung astronomischer Phänomene mit bloßem Auge ist die älteste Art der Beobachtung. Der historisch nächste Schritt war die Entwicklung von optischen Instrumenten zur Vergrößerung sehr weit entfernter Objekte und zur Verstärkung des schwachen Lichteinfalls. Eine weitere Verbesserung in der astronomischen Beobachtung stellen verbesserte Aufnahmemethoden dar. Waren dies zu Beginn noch chemisch beschichtete Photoplatten, die sich bei Lichteinfall ausreichender Wellenlänge dunkel verfärbten, so geschieht das Aufnehmen eines Bildes heute mittels sogenannter CCDs (Charged Coupled Devices), in denen Lichteinfall mittels Halbleiterelementen detektiert wird. \\
In diesem Versuch sollen einige Objekte mittels eines kleinen Spiegelteleskops beobachtet werden, sowie eine Galaxie mittels des 50 cm-Teleskops beobachtet und mittels Filteraufnahmen graphisch aufgewertet werden.  