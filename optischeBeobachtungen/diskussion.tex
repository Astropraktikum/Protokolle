\section{Diskussion}
Bei sowohl den Aufnahmen mit dem Linsenteleskop als auch mit den Spiegelteleskopen in den Kuppeln wurden die Einschränkungen bei der Auflösung bedingt durch das Seeing deutlich. Dieser Effekt kann theoretisch durch die Beobachtung eines als annähernd punktförmig wahrgenommenen Sternes ausgeglichen werden: Man errechnet aus dem beobachteten Flimmern des Sterns eine Korrektur, die dann auf die eigentliche Beobachtung angewendet wird. Da dies in Echtzeit geschehen muss, ist dieses Verfahren, das sich adaptive Optik nennt, allerdings sehr aufwendig und steht hier nicht zur Verfügung. \\
Des weiteren könnte mit einem größerem Teleskop eine größere Lichtausbeute erreicht werden, sodass auch lichtschwächere Objekte sichtbar sind. Dies ist aber auch mit hohen Kosten verbunden. 


