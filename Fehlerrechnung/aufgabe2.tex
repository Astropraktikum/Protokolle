\documentclass[titlepage]{scrartcl}
% Code Darstellung
\usepackage{listings}
\usepackage{listingsutf8}
\usepackage{multicol}

%lange Tabellen
\usepackage{longtable}
%Referenzen zwischen unterschiedlichen Dateien
\usepackage{xr}
\externaldocument{theorie}
\usepackage{lscape}
%Deutsche Sprachunterstützung
\usepackage[utf8]{inputenc}
\usepackage[ngerman]{babel}
\usepackage{marvosym}
\DeclareUnicodeCharacter{20AC}{\EUR}

%Für das Einbinden von Bildern
\usepackage{graphicx}

%Tabellen
\usepackage{array}

%Tabellen automatisch schoener
\usepackage{booktabs}

%Caption
\usepackage{caption}
\usepackage{subcaption}

%Formeln
\usepackage{mathtools}
\usepackage{amsmath}
\usepackage{amssymb}
\usepackage{amstext}
\usepackage{dsfont}

%\usepackage{mnsymbol}

% Interssante natbib Optionen: 
% numbers : Nummerierte Zitateinheiten
% sort&compress : Bei mehrfachen Zitaten, Sortierung und ggf. Verkürzungen
%\usepackage[]{natbib}

%Vectorpfeile schöner
\usepackage{esvect}

%Formatierung
\usepackage[T1]{fontenc}
\usepackage{lmodern}
\usepackage{microtype}

%Schaltbilder malen
%\usepackage[europeanresistors,cuteinductors,siunitx]{circuitikz}
\usepackage{comment}
\usepackage{csquotes}

%Formatierungsanweisungen
\newcommand{\wichtig}[1]{\underline{\large{#1}}}
\newcommand{\aref}[1]{Abb. \ref{#1}}
\newcommand{\R}{\mathbb{R}}
\newcommand{\K}{\mathbb{K}}
\newcommand{\C}{\mathbb{C}}

%Klickbare Referenzen
%\usepackage[hidelinks]{hyperref}


\begin{document}
\section{Aufgabe 2} 
\subsection{a)} 
Die erste Messung werde mit $P_{orb, 1}$ bezeichnet, die zweite mit $P_{orb,2}$ bezeichnet. 
Es gilt 
\begin{equation}
\Delta P_{orb} = P_{orb, 1} - P_{orb, 2} = 24.31704 d - 24.316 d = (0.001 \pm 0.002) d.
\end{equation} 

Für den Fehler ergibt sich: 
\begin{equation}
\Delta (\Delta P_{orb}) = \sqrt{(\frac{\partial \Delta P_{orb}}{\partial P_{orb, 1}})^2 \cdot (\Delta P_{orb, 1})^2 + (\frac{\partial \Delta P_{orb}}{\partial P_{orb, 2}})^2 \cdot (\Delta P_{orb, 2})^2} = \sqrt{(\Delta P_{orb, 1})^2 + (\Delta P_{orb, 2})^2} = 0.002 d
\end{equation}

subsection{b)}
Diese Messung von 2010 sei mit $P_{orb, 1}$ bezeichnet. 
Analog ergibt sich: 

\begin{equation}
\Delta P_{orb} = P_{orb, 1} - P_{orb, 3} = 24.31704 d - 24.31617 d = (0.000870 \pm 0.000093) d.
\end{equation}

Für den Fehler ergibt sich: 
\begin{equation}
\Delta (\Delta P_{orb}) = \sqrt{(\frac{\partial \Delta P_{orb}}{\partial P_{orb, 1}})^2 \cdot (\Delta P_{orb, 1})^2 + (\frac{\partial \Delta P_{orb}}{\partial P_{orb, 3}})^2 \cdot (\Delta P_{orb, 3})^2} = \sqrt{(\Delta P_{orb, 1})^2 + (\Delta P_{orb, 3})^2} = 0.000093 d
\end{equation}

\section{Aufgabe 3}
\subsection{a)}
Es ist ein $\kappa$ gesucht, sodass
\begin{equation}
\int_{-\infty}^{\infty} \kappa \cdot exp(-\frac{(x-M)^2}{2\sigma^2}) dx = 1.
\end{equation}

Es gilt:
\begin{equation}
\int_{-\infty} ^{\infty} \kappa \cdot exp(-\frac{(x-M)^2}{2\sigma^2}) dx = \int_{-\infty} ^{\infty} \kappa \cdot exp(-\frac{(x-M)^2}{2\sigma^2}) d(x-M) = \kappa \cdot \sqrt {2\pi} \cdot \sigma = 1. 
\end{equation}

Daraus ergibt sich: 
\begin{equation}
\kappa \cdot \sqrt{2\pi} \cdot \sigma = 1
\end{equation} oder


\begin{equation}
\kappa = \frac{1}{\sqrt{2\cdot \pi} \cdot \sigma}
\end{equation}
\subsection{b)}
Da in diesem Fall n = 1, folgt aus dem angegebenen Fehler bei $P_{orb, 1} = (24.31704 \pm 0.00006) d$ eine Standardabweichung von ebenfalls $0.00006 d$. \\
Die Wahrscheinlichkeit $\beta$, dass sich ein Messwert innerhalb des Intervalls $ (M-\Delta. M+ \Delta)$ befindet, beträgt also 

\begin{equation}
\beta = \int_{M - \Delta}^{M + \Delta} f(x) dx, 
\end{equation}
wobei 
\begin{equation}
f(x) = \frac{1}{\sqrt{2\cdot \pi} \cdot \sigma} \cdot exp(-\frac{(x-M)}{2\sigma^2}). 
\end{equation}








\end{document}