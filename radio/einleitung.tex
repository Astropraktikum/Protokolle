\section{Einleitung}
Der Großteil astronomischen Wissens stammt aus der Beobachtung elektromagnetischer Wellen. Dabei war die Beobachtung bis 1930 auf das Spektrum des sichtbaren Lichts beschränkt. Karl Guthe Jansky wird als Entdecker der Radioastronomie angesehen, weil er als erster das Zentrum der Milchstraße und die Sonne als Radiostörquellen ausmachte.\footnote{\cite{sky}} Heute wird die Radioastronomie genutzt um Quasare, Seyfertgalaxien und die 21 cm-Linie von neutralem Wasserstoff zu messen.
