\section{Einleitung}
Astronomische Beobachtungen finden nicht nur im optischen Bereich statt, auch im Bereich der Radiowellen, Infrarotstrahlung oder Röntgenstrahlung können Erkenntnisse gewonnen werden. Bis 1930 war die Beobachtung jedoch auf das Spektrum des sichtbaren Lichts beschränkt, bis Karl Guthe Jansky bei der Messung von elektromagnetischer Strahlung bei Gewittern Radiostrahlung aus dem Zentrum der Milchstraße entdeckte.\footnote{\cite{sky}} In den folgenden Jahrzehnten wurden viele weitere Entdeckungen, wie z.B. Pulsare oder die 3 K - Hintergrundstrahlung, gemacht. Mithilfe der 21 cm-Wasserstofflinie, die in interstellaren Gaswolken entsteht, kann z.B. auch die Rotationsgeschwindigkeit der Milchstraße gemessen werden. Auch heute ist die Radioastronomie ein wichtiges Teilgebiet der Astronomie, in dem zurzeit große Anstrengungen unternommen werden, um bessere Anlagen zu konstruieren. 