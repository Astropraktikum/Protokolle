\documentclass[titlepage]{scrartcl}
% Code Darstellung
\usepackage{listings}
\usepackage{listingsutf8}
\usepackage{multicol}

%lange Tabellen
\usepackage{longtable}
%Referenzen zwischen unterschiedlichen Dateien
\usepackage{xr}
%\externaldocument{theorie}
\usepackage{lscape}
%Deutsche Sprachunterstützung
\usepackage[utf8]{inputenc}
\usepackage[ngerman]{babel}
\usepackage{marvosym}
\DeclareUnicodeCharacter{20AC}{\EUR}

%Für das Einbinden von Bildern
\usepackage{graphicx}

%Tabellen
\usepackage{array}

%Tabellen automatisch schoener
\usepackage{booktabs}

%Caption
\usepackage{caption}
\usepackage{subcaption}

%Formeln
\usepackage{mathtools}
\usepackage{amsmath}
\usepackage{amssymb}
\usepackage{amstext}
\usepackage{dsfont}

%\usepackage{mnsymbol}

% Interssante natbib Optionen: 
% numbers : Nummerierte Zitateinheiten
% sort&compress : Bei mehrfachen Zitaten, Sortierung und ggf. Verkürzungen
%\usepackage[]{natbib}

%Vectorpfeile schöner
\usepackage{esvect}

%Formatierung
\usepackage[T1]{fontenc}
\usepackage{lmodern}
\usepackage{microtype}

%Schaltbilder malen
%\usepackage[europeanresistors,cuteinductors,siunitx]{circuitikz}
\usepackage{comment}
\usepackage{csquotes}

%Formatierungsanweisungen
\newcommand{\wichtig}[1]{\underline{\large{#1}}}
\newcommand{\aref}[1]{Abb. \ref{#1}}
\newcommand{\R}{\mathbb{R}}
\newcommand{\K}{\mathbb{K}}
\newcommand{\C}{\mathbb{C}}

%Klickbare Referenzen
%\usepackage[hidelinks]{hyperref}

\begin{document}
F"ur die Frequenzverteilung der Leistung ergibt sich:
\begin{equation}
P(f) = \frac{2 \, \mathrm{W}}{\sigma \cdot \sqrt{2 \cdot \pi}} \cdot \exp\left({\frac{-(f-f_0)^2}{2 \cdot \sigma^2}}\right), 
\end{equation}
wobei $f_0 = 900\,  \mathrm{MHz}$ und $\sigma = 40\,  \mathrm{kHz}$. Damit ergibt sich f"ur $ f = 900\,  \mathrm{MHz} $ eine Leistung von $ P = \frac{2\,  \mathrm{W}}{\sqrt{2 \pi} \cdot 40 \mathrm{kHz}} \approx 0.020 \frac{W}{kHz}$. 
Da der Mond von der Erde etwa 375000 km entfernt ist, ergibt sich f"ur den Strahlungsfluss auf der Erde: 
\begin{equation}
S = \frac{P(Erde)}{A} = \frac{\frac{2\,  \mathrm{W}}{\sqrt{2 \pi} \cdot 40\,  \mathrm{kHz}}}{4 \cdot r_{\mathrm{Erde-Mond}}^2 \pi} \approx 1.13 \cdot 10^{-20} \frac{\mathrm{W}}{\mathrm{kHz}\,  \mathrm{m}^2} = 1.13 \cdot 10^6 \, \mathrm{Jy}. 
\end{equation}
F"ur die Intensit"aten 
\begin{equation}
S_{\nu} \approx \nu^{-\alpha}. 
\end{equation}
Somit ergibt sich: 
\begin{equation}
S_{\nu_1} = S_{\nu_0} \cdot (\frac{\nu_1}{\nu_0})^\alpha, 
\end{equation}
und somit:

\begin{equation}
S_{900 \, \mathrm{MHz}}(\mathrm{Cas\ A}) = S_{1400 \, \mathrm{MHz}}(\mathrm{Cas\ A}) \cdot \left(\frac{900}{1400}\right)^{-0.75} \approx 3343\,  \mathrm{Jy}
\end{equation} und analog:

\begin{equation}
S_{900 \, \mathrm{MHz}}(\mathrm{Cyg\ A}) \approx 2136\,  \mathrm{Jy}
\end{equation} und 

\begin{equation}
S_{900 \, \mathrm{MHz}}(\mathrm{Tau\ A}) \approx 913\, \mathrm{Jy}.
\end{equation}
Diese Werte sind etwa drei Gr"o"senordnungen kleiner als der Strahlungsfluss des Handys auf dem Mond. 

\end{document}