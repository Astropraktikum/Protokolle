\section{Diskussion}
Sämtliche Messungen weisen eine durch die begrenzte Auflösungsfähigkeit des Teleskops (etwa 6 $^\circ$) bedingte Ungenauigkeit auf. Somit könnten durch eine verbesserte Auflösung des Teleskops deutlich bessere Ergebnisse bei etwa dem npoint-Scan der Sonne und damit auch genauere Werte für den Ort der Sonne am Himmel bestimmt werden. Weiter wäre es unter Umständen möglich, etwa Phänomene wie Sonnenflecken auch im Radiobereich zu beobachten. \\
Problematisch ist allerdings, dass aufgrund der hohen Wellenlänge der Radiostrahlen eine sehr viel größere Teleskopfläche notwendig wäre, um  deutlich verbesserte Genauigkeit zu erreichen. Dies ist selbstredend mit hohem finanziellen Aufwand verbunden und deswegen schwierig. \\
Eine weitere Quelle für Fehler war insbesondere beim Scan der Milchstraße die gegenseitige Beeinflussung beider Radioteleskope, welche sich als extremer Peak in den Daten darstellt. Dieser Fehler könnte vermieden werden, wenn die Radioteleskope nicht parallel sondern bloß einzeln betrieben werden würden. Weiter könnte eine Abschirmung der beiden Teleskope gegeneinander die Störung verhindern, was sich allerdings aufgrund Effekte wie Beugung der Radiowellen, deren Wellenlänge bekanntlich im cm- bis m-Bereich liegt, ebenfalls schwierig ist. 