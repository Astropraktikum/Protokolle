\section{Einleitung}
Wenn die Navigation auf hoher See heute vor allem mittels GPS geschieht, so wurde früher die Sonne zur Positionsbestimmung verwendet. Dies geschah mit einem Sextanten, mit dem der Winkelabstand zwischen dem Horizont und einem astronomischen Objekt bestimmt werden kann. Im Falle der Seefahrt wurde so die Höhe der Sonne über dem Horizont bestimmt, sodass mit Kenntnis der exakten Uhrzeit eine Abschätzung der eigenen Position möglich ist. In dieser Messung soll aber nicht die Deklination eines Objekts, sondern dessen Azimut, also der Winkel relativ zur Nord-Süd-Richtung, bestimmt werden. Das hier angepeilte Objekt ist ein etwa 13 km entfernter Fensehsender auf dem Geisberg. Dazu wird ein Theodolit verwendet, mittels dem die Azimutwerte von Turm und Sonne relativ zu einem Nullpunkt, dessen absolute Ausrichtung nicht bestimmt werden kann, bestimmt werden. Aus Tabellen kann die Position der Sonne berechnet werden, sodass am Ende eine Aussage über den absoluten Azimut des Fernsehturms möglich ist. 
