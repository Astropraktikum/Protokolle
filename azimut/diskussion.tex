\section{Diskussion}
Aufgrund der Präzision, mit der die Werte für den relativen Azimut abgelesen werden können, die sich nach Aussage des Betreuers auf etwa eine Bogensekunde beläuft, ist eine sehr genaue Messung des Azimuts möglich. Die Genauigkeit dieser Messung zeigt sich etwa im geringen Fehler bei der Messung des relativen Turmazimuts, der sich auf lediglich 0.4 Bogensekunden beläuft. Einen weit größeren Fehler liefert die Bestimmung des Nullpunkts der Skala, in den effektiv der gemessene und der durch Rechnung bestimmte absolute Sonnenazimut einfließen. Da für den gemessenen Sonnenazimut das Überstreichen der im Fernrohr sehr ausgedehnten Sonnenscheibe betrachtet wird und das Überstreichen des Zentrums mittels eines händischen Zeitmessung per Stoppuhr durchgeführt wird, ist ein größerer Fehler zu erwarten. Weitere Fehler können sich etwa durch die Interpolation von Sonnenazimut und -deklination aus Tabellenwerten ergeben. Da aber die Rotation der Erde um die Sonne in sehr guter Näherung gleichförmig ist, ist eine lineare Interpolation gerechtfertigt. \\
Vergleicht man diese Methode etwa mit der Alternative, den Azimut mittels eines Kompass, der die Nord-Süd-Richtung liefert, zu bestimmen, so treten hier viele Schwierigkeiten wie etwa lokale Änderungen des Magnetfeldes, die Verschiebung zwischen magnetischem und geographischen Pol, etc. auf. \\
Betrachtet man die Umgebung von Bamberg auf google maps, so erkennt man, dass der Fernsehturm ziemlich exakt im Osten liegt, was mit der Messung zumindest im Groben übereinstimmt. \\
Eine Verbesserung der Messung wäre möglich, wenn die Bestimmung des relativen Sonnenazimuts verbessert werden könnte. Dies wäre etwa durch das Anschließen einer Kamera und eines Rechners an das Fernrohr des Theodoliten möglich, sodass das Überstreichen der Sonnenränder über das Fadenkreuz exakter mit einer bildverarbeitenden Software möglich wäre. Weiter könnte die Messung durch eine bessere horizontale Ausrichtung des Theodoliten erreicht werden. Da bei der Beobachtung auffiel, dass etwa die Ausdehnung des Sockels und der Montierung bei Erwärmung die horizontale Ausrichtung beeinflussen, könnte dieser Fehler durch eine konstante Temperierung der angesprochenen Elemente verringert werden. \\
Tatsächlich weißt die Messung aber eine sehr große Genauigkeit auf. 